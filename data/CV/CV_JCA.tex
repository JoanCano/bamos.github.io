%% start of file `template.tex'.
%% Copyright 2006-2013 Xavier Danaux (xdanaux@gmail.com).
%
% This work may be distributed and/or modified under the
% conditions of the LaTeX Project Public License version 1.3c,
% available at http://www.latex-project.org/lppl/.

\documentclass{my_cv}

% adjust the page margins
%\setlength{\hintscolumnwidth}{3cm}                % if you want to change the width of the column with the dates
%\setlength{\makecvtitlenamewidth}{10cm}           % for the 'classic' style, if you want to force the width allocated to your name and avoid line breaks. be careful though, the length is normally calculated to avoid any overlap with your personal info; use this at your own typographical risks...

% personal data
%\extrainfo{additional information}                 % optional, remove / comment the line if not wanted
%photo[64pt][0.4pt]{picture}                       % optional, remove / comment the line if not wanted; '64pt' is the height the picture must be resized to, 0.4pt is the thickness of the frame around it (put it to 0pt for no frame) and 'picture' is the name of the picture file

%\quote{Seeking a full time position in software development }

% to show numerical labels in the bibliography (default is to show no labels); only useful if you make citations in your resume
%\makeatletter
%\renewcommand*{\bibliographyitemlabel}{\@biblabel{\arabic{enumiv}}}
%\makeatother
%\renewcommand*{\bibliographyitemlabel}{[\arabic{enumiv}]}% CONSIDER REPLACING THE ABOVE BY THIS

% bibliography with mutiple entries
%\usepackage{multibib}
%\newcites{book,misc}{{Books},{Others}}
%----------------------------------------------------------------------------------
%            content
%----------------------------------------------------------------------------------
\begin{document}
%\begin{CJK*}{UTF8}{gbsn}
%-----       resume       ---------------------------------------------------------
\makecvtitle
\section{Formación \faBook}

\cventry{2016--2017}{ Sistemas de Información Geográfica y Teledetección}{Máster en TIG}{Universidad de Zaragoza}{}{}

\cventry{2012--2016}{ y Ordenación del Territorio}{ Grado en Geografía}{Universidad de Alicante}{}{}


\section{Experiencia laboral \faSuitcase}
\cventry{Julio 1017 - Actual}{Técnico en tecnologías geospaciales}{3DSCANNER}{Zaragoza}{}{ Desarrollo de trabajos de topografía, escaneado y fotogrametría en el ámbito del Patrimonio, Medio Ambiente, Ingeniería e Industria. Soporte técnico de Leica en las soluciones HDS. Soporte técnico oficial de drones de SenseFly, Parrot y Flyability.}

\cventry{Junio 2019}{Instructor}{Fundación laboral de la construcción}{Zaragoza}{}{Fotografía aérea y fotogrametría para construcción y obra civil con el uso de drones}

\cventry{Febrero 2016 - Mayo 2016}{Prácticas de empresa en el Ayuntamiento de Sella y Laboratorio de Geomática}{Universidad de Alicante}{Alicante}{}{Diseño de bases de datos para la gestión del cementerio municipal. Infracciones urbanísticas. Cartografía municipal}



\section{Formación: Cursos y seminarios  \faBook}
\cvitemwithcomment{Bases de datos espaciales}{PostGis}{2015}
\cvitemwithcomment{Microviñas}{una herramienta para el ecosistema empresarial rural}{2017}
\cvitemwithcomment{Curso avanzado de Piloto de Drones por el Real Aeroclub de Zaragoza}{60 horas}{2017}
\cvitemwithcomment{Curso Teórico y Práctico de aeronave pilotada por control remoto de clase avión}{eBee}{2017}
\cvitemwithcomment{Curso de manejo de Pix4D}{10 horas}{2017}
\cvitemwithcomment{SenseFly Trainer Certificate en eBeeX}{}{2017}
\cvitemwithcomment{Jornada sobre el uso de drones y escáner 3D para La Junta de Castilla y León}{}{2018}
\cvitemwithcomment{Jornada sobre el uso de drones y escáner 3D aplicada a PRL para Asepeyo}{}{2018}
\cvitemwithcomment{Curso de Blender}{Postprocesado fotogramétrico y reconstrucción virtual. 60h}{2019}
\cvitemwithcomment{Comunicación oral en ISMAR10: Monitorización de los recursos hidrológicos nivales}
{}{}
\cvitemwithcomment{}{el glaciar de Monte Perdido (Huesca)}{2019}
\cvitemwithcomment{Curso de instructor de RPAS, por EPRYD}{Escuela ULM número de registro 2018004169. 15 horas}{2019}
\cvitemwithcomment{Formación Leica escáner láser 3D}{20h}{2019}


\section{Formación: Premios y Becas \faBook}
\cvitemwithcomment{Mención de excelencia en el Trabajo de Fin de Máster}{}{}
\cvitemwithcomment{}{Calibración de sensores multiespectrales y su aplicación en UAV. Caso de estudio: Análisis de la cámara Sequoia}{}
\cvitemwithcomment{}{}{2018}


\section{Formación: Voluntariado \faBook}
\cvitemwithcomment{Voluntariado en Cobán (Guatemala)}{}{}
\cvitemwithcomment{}{proyecto de cartografía colaborativa basado en OpenStreetMap a través de la Universidad de Alicante}{2016}
\cvitemwithcomment{Miembro del grupo DroneMapZ}{proyecto ciudadano en las Convocatorias Cesar de Etopia (Zaragoza)
}{2018}
\cvitemwithcomment{Miembro activo de Mapeado Colaborativo}{grupo residentes en Zaragoza Activa (Zaragoza)}{2017 - actual}

\section{Formación: Idiomas \faBook}
\cvitemwithcomment{Inglés}{nivel B1-PET}{}
\cvitemwithcomment{Castellano}{Alto}{}
\cvitemwithcomment{Catalán}{Alto}{Lengua materna}

\section{Principales trabajos realizados \faCogs}
\cventry{2019}{Levantamiento topográfico y delineación 3D}{Municipios de Artiès, Baquèira, Salardú y tresdós en el Vall d’Aran}{}{}{}

\cventry{2019}{Realizando trabajos de enlace con la geodesia espacial,  escaneado láser 3D y drones}{Documentación geométrica del Dolmen de Guadalperal y Agustóbriga}{}{}{}

\cventry{2019}{Realizando trabajos de enlace con la geodesia espacial, escaneado láser 3D, fotogrametría aérea y terrestre}{Documentación geométrica, espacial y multiespectral del Castillo de Santed}{}{}{}

\cventry{2019}{Realizado con eBee RTK en Moneva y Moyuela}{Levantamiento topográfico para subestación eléctrica y nuevas vías de transporte}{}{}{}

\cventry{2019}{Análisis de voladuras en minas subterráneas}{Inspección con dron Elios 2}{}{}{}

\cventry{2019}{Realizando trabajos de enlace con la geodesia espacial, poligonal y escaneado láser 3D}{Documentación geométrica de Polidux (Monzón)}{}{}{}

\cventry{2019}{Realizando trabajos de enlace con la geodesia espacial, poligonal y escaneado láser 3D}{Documentación geométrica de la Fuente del Berro (Madrid)}{}{}{}

\cventry{2019}{Realizando trabajos de enlace con la geodesia espacial, poligonal y escaneado láser 3D}{Documentación geométrica de Necrópolis Musulmana en Tudela (Navarra)}{}{}{}

\cventry{2019}{Realizando trabajos de enlace con la geodesia espacial, poligonal y escaneado láser 3D}{Documentación geométrica de las instalaciones del FIDAMC (Madrid)}{}{}{}

\cventry{2018}{Realizando trabajos de nivelación geométrica, análisis de estructuras y asentamiento de edificios}{Auscultación del Colector de aguas fecales de la Calle Santa Coloma de Barcelona}{}{}{}

\cventry{2018}{Reconstrucción tridimensional de una manyatta maasai para documental mediante realidad virtual}{Fotogrametría y escaneado láser de un escenario de la serie “La Peste”}{}{}{}

\cventry{2018}{Realizando trabajos de enlace con la geodesia espacial, poligonal y escaneado láser 3D}{Documentación geométrica de Viajes de agua de Amaniel (Madrid)}{}{}{}

\cventry{2017}{Realizando trabajos de enlace con la geodesia espacial, poligonal y escaneado láser 3D}{Relevé 3D de certains secteurs de la grotte de Bruniquel (Francia)}{}{}{}

\cventry{2017}{Realizando trabajos de enlace con la geodesia espacial, poligonal y escaneado láser 3D}{Documentación geométrica de la Torre de Conchel (Huesca)}{}{}{}




\section{Habilidades \faBullseye}
%\cvlistdoubleitem{Ruby/Rails}{Linux}
\cvlistdoubleitem{Sistemas de Información Geográfica}{Teledetección}
\cvlistdoubleitem{Piloto de drones}{Webmapping}
\cvlistdoubleitem{Fotogrametría}{Láser Escáner}
\cvlistdoubleitem{Edición 3D}{Cartografía}
\cvlistdoubleitem{Linux, Windows}{Git}

% \cvlistdoubleitem{\LaTeX}{Swift (iOS Development)}

%\section{Familiar With \faPlusSign}
%\cvlistdoubleitem{Adobe Illustrator}{CSS/SASS}

%\clearpage\end{CJK*}                              % if you are typesetting your resume in Chinese using CJK; the \clearpage is required for fancyhdr to work correctly with CJK, though it kills the page numbering by making \lastpage undefined
\end{document}
